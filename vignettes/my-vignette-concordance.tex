\Sconcordance{concordance:my-vignette.tex:my-vignette.Rnw:%
1 12 1 49 0 6 1 2 0 7 1 16 0 9 1 17 0 9 1 4 0 5 1 4 0 5 1 4 0 2 1}
